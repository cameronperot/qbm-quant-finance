% Misc. commands
\newcommand{\CNOT}[2]{\text{C}_{#1}\text{NOT}_{#2}}
\newcommand{\SWAP}{\text{SWAP}}
\newcommand{\CZ}{\text{CZ}}
\newcommand{\tr}{\text{tr}}
\newcommand{\frt}{\frac{1}{\sqrt{2}}}
\newcommand{\Z}{\mathbb{Z}}
\newcommand{\R}{\mathbb{R}}
\newcommand{\N}{\mathbb{N}}
\newcommand{\mat}[1]{\mathbf{#1}}
\newcommand{\binset}{\{0,1\}}
\newcommand{\spinset}{\{+1,-1\}}
\newcommand{\spause}{s_{\text{pause}}}
\newcommand{\tpause}{t_{\text{pause}}}
\newcommand{\Deltapause}{\Delta_{\text{pause}}}
\newcommand{\Deltaquench}{\Delta_{\text{quench}}}
\newcommand{\alphaquench}{\alpha_{\text{quench}}}
\newcommand{\squench}{s_{\text{quench}}}
\newcommand{\tquench}{t_{\text{quench}}}
\newcommand{\ptheory}{p_{\text{theory}}}
\newcommand{\pdata}{p_{\text{data}}}
\newcommand{\psamples}{p_{\text{samples}}}
\newcommand{\pmodel}{p_{\text{model}}}
\newcommand{\betahat}{\hat{\beta}}
\newcommand{\rcs}{\gamma_{\text{relative}}}
\newcommand{\trelative}{t_{\text{relative}}}
\DeclarePairedDelimiterX{\infdivx}[2]{(}{)}{#1\;\delimsize\|\;#2}
\newcommand{\DKL}{D_{\text{KL}}\infdivx}
\renewcommand{\vec}[1]{\mathbf{#1}}
\newcommand{\basistwo}{\{\ket{00}, \ket{01}, \ket{10}, \ket{11}\}}
\DeclarePairedDelimiter{\norm}{\lVert}{\rVert}
\DeclarePairedDelimiter{\abs}{\lvert}{\rvert}
\DeclareMathOperator*{\argmax}{arg\,max}
\DeclareMathOperator*{\argmin}{arg\,min}

% Identity operator (on K^2)
\newcommand{\idtwo}{
    \begin{bmatrix}
        1 & 0 \\
        0 & 1
    \end{bmatrix}
}

% Hadamard operator
\newcommand{\Hgate}{
    \frac{1}{\sqrt{2}} \begin{bmatrix}
        1 & 1 \\
        1 & -1
    \end{bmatrix}
}

% T gate
\newcommand{\Tgate}{
    \begin{bmatrix}
        1 & 0 \\
        0 & e^{i \pi / 4}
    \end{bmatrix}
}

% T gate dag
\newcommand{\Tgatedag}{
    \begin{bmatrix}
        1 & 0 \\
        0 & e^{-i \pi / 4}
    \end{bmatrix}
}

% R_X
\newcommand{\RXGate}[1]{
    \begin{bmatrix}
        \cos\frac{#1}{2} & -i\sin\frac{#1}{2} \\
        -i\sin\frac{#1}{2} & \cos\frac{#1}{2}
    \end{bmatrix}
}

% R_Y
\newcommand{\RYGate}[1]{
    \begin{bmatrix}
        \cos\frac{#1}{2} & -\sin\frac{#1}{2} \\
        \sin\frac{#1}{2} & \cos\frac{#1}{2}
    \end{bmatrix}
}

% R_Z
\newcommand{\RZGate}[1]{
    \begin{bmatrix}
        e^{-i{#1}/2} & 0 \\
        0 & e^{i{#1}/2}
    \end{bmatrix}
}

% Pauli X operator
\newcommand{\XGate}{
    \begin{bmatrix}
        0 & 1 \\
        1 & 0
    \end{bmatrix}
}

% Pauli Y operator
\newcommand{\YGate}{
    \begin{bmatrix}
        0 & -i \\
        i & 0
    \end{bmatrix}
}

% Pauli Z operator
\newcommand{\ZGate}{
    \begin{bmatrix}
        1 & 0 \\
        0 & -1
    \end{bmatrix}
}

% The |0⟩ state in vector form
\newcommand{\zerostate}{
    \begin{bmatrix}
        1 \\
        0
    \end{bmatrix}
}

% The |1⟩ state in vector form
\newcommand{\onestate}{
    \begin{bmatrix}
        0 \\
        1
    \end{bmatrix}
}

% The |-⟩ state in vector form
\newcommand{\minusstatey}{
    \frac{1}{\sqrt{2}}
    \begin{bmatrix}
        1 \\
        -i
    \end{bmatrix}
}

% The |+⟩ state in vector form
\newcommand{\plusstatey}{
    \frac{1}{\sqrt{2}}
    \begin{bmatrix}
        1 \\
        i
    \end{bmatrix}
}

% The |-'⟩ state in vector form
\newcommand{\minusstate}{
    \frac{1}{\sqrt{2}}
    \begin{bmatrix}
        1 \\
        -1
    \end{bmatrix}
}

% The |+'⟩ state in vector form
\newcommand{\plusstate}{
    \frac{1}{\sqrt{2}}
    \begin{bmatrix}
        1 \\
        1
    \end{bmatrix}
}

% Transpose
\makeatletter
\newcommand*{\transpose}{%
  {\mathpalette\@transpose{}}%
}
\newcommand*{\@transpose}[2]{%
  % #1: math style
  % #2: unused
  \raisebox{\depth}{$\m@th#1\intercal$}%
}
\makeatother
\newcommand{\T}{^{\transpose}}
