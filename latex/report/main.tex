\documentclass[11pt,twoside]{report}
\usepackage[a4paper,width=150mm,top=25mm,bottom=25mm,bindingoffset=6mm]{geometry}
\usepackage[utf8]{inputenc}
\usepackage{hyperref}
\hypersetup{
    colorlinks,
    citecolor=black,
    filecolor=black,
    linkcolor=black,
    urlcolor=black,
    breaklinks=true
}
\usepackage{xstring}
\usepackage{adjustbox}
\usepackage{algorithm}
\usepackage{algpseudocode}
\usepackage{amssymb}
\usepackage{amsmath}
\usepackage{bbm}
\usepackage{bm}
\usepackage{booktabs}
\usepackage[style=numeric-comp, sorting=none]{biblatex}
\usepackage{braket}
\usepackage{catchfile}
\usepackage[capitalize]{cleveref}
\usepackage{color}
\usepackage{datatool}
\usepackage{easy-todo}
\usepackage{float}
\usepackage{sourcecodepro}
\usepackage[T1]{fontenc}
\usepackage{graphicx}
\usepackage{listings}
\usepackage{multirow}
\usepackage{mathtools}
\usepackage{notoccite}
\usepackage{siunitx}
\usepackage{tabularx}
\usepackage{pythonhighlight}


\addbibresource{../references.bib}
\graphicspath{ {../images/}{../../results/plots/} }
\lstset{basicstyle=\small\ttfamily}

% Misc. commands
\newcommand{\CNOT}[2]{\text{C}_{#1}\text{NOT}_{#2}}
\newcommand{\SWAP}{\text{SWAP}}
\newcommand{\CZ}{\text{CZ}}
\newcommand{\tr}{\text{tr}}
\newcommand{\frt}{\frac{1}{\sqrt{2}}}
\newcommand{\vv}{\mathbf{v}}
\newcommand{\hv}{\mathbf{h}}
\newcommand{\basistwo}{\{\ket{00}, \ket{01}, \ket{10}, \ket{11}\}}
\DeclarePairedDelimiter{\norm}{\lVert}{\rVert}
\DeclarePairedDelimiter{\abs}{\lvert}{\rvert}

% Identity operator (on K^2)
\newcommand{\idtwo}{
    \begin{bmatrix}
        1 & 0 \\
        0 & 1
    \end{bmatrix}
}

% Hadamard operator
\newcommand{\Hgate}{
    \frac{1}{\sqrt{2}} \begin{bmatrix}
        1 & 1 \\
        1 & -1
    \end{bmatrix}
}

% T gate
\newcommand{\Tgate}{
    \begin{bmatrix}
        1 & 0 \\
        0 & e^{i \pi / 4}
    \end{bmatrix}
}

% T gate dag
\newcommand{\Tgatedag}{
    \begin{bmatrix}
        1 & 0 \\
        0 & e^{-i \pi / 4}
    \end{bmatrix}
}

% R_X
\newcommand{\RX}[1]{
    \begin{bmatrix}
        \cos\frac{#1}{2} & -i\sin\frac{#1}{2} \\
        -i\sin\frac{#1}{2} & \cos\frac{#1}{2}
    \end{bmatrix}
}

% R_Y
\newcommand{\RY}[1]{
    \begin{bmatrix}
        \cos\frac{#1}{2} & -\sin\frac{#1}{2} \\
        \sin\frac{#1}{2} & \cos\frac{#1}{2}
    \end{bmatrix}
}

% R_Z
\newcommand{\RZ}[1]{
    \begin{bmatrix}
        e^{-i{#1}/2} & 0 \\
        0 & e^{i{#1}/2}
    \end{bmatrix}
}

% Pauli X operator
\newcommand{\X}{
    \begin{bmatrix}
        0 & 1 \\
        1 & 0
    \end{bmatrix}
}

% Pauli Y operator
\newcommand{\Y}{
    \begin{bmatrix}
        0 & -i \\
        i & 0
    \end{bmatrix}
}

% Pauli Z operator
\newcommand{\Z}{
    \begin{bmatrix}
        1 & 0 \\
        0 & -1
    \end{bmatrix}
}

% The |0⟩ state in vector form
\newcommand{\zerostate}{
    \begin{bmatrix}
        1 \\
        0
    \end{bmatrix}
}

% The |1⟩ state in vector form
\newcommand{\onestate}{
    \begin{bmatrix}
        0 \\
        1
    \end{bmatrix}
}

% The |-⟩ state in vector form
\newcommand{\minusstatey}{
    \frac{1}{\sqrt{2}}
    \begin{bmatrix}
        1 \\
        -i
    \end{bmatrix}
}

% The |+⟩ state in vector form
\newcommand{\plusstatey}{
    \frac{1}{\sqrt{2}}
    \begin{bmatrix}
        1 \\
        i
    \end{bmatrix}
}

% The |-'⟩ state in vector form
\newcommand{\minusstate}{
    \frac{1}{\sqrt{2}}
    \begin{bmatrix}
        1 \\
        -1
    \end{bmatrix}
}

% The |+'⟩ state in vector form
\newcommand{\plusstate}{
    \frac{1}{\sqrt{2}}
    \begin{bmatrix}
        1 \\
        1
    \end{bmatrix}
}


%----------------------------------------------------------------------------------------
% Title Page
%----------------------------------------------------------------------------------------
\pagenumbering{gobble}
\title{
    {Quantum Boltzmann Machines}\\
    {\large Applications in Quantitative Finance}
}
\author{
    {\LARGE Cameron Perot\vspace{1cm}}\\
    {Master's Thesis\vspace{0.1cm}}\\
    {\small submitted to\vspace{0.1cm}}\\
    {the Faculty of Mathematics, Computer Science, and Natural Sciences}\\
    {of RWTH Aachen University\vspace{0.1cm}}\\
    {\small written at\vspace{0.1cm}}\\
    {Jülich Supercomputing Centre}\\
    {Forschungszentrum Jülich\vspace{1cm}}\\
    {First Examiner: Prof. Dr. Kristel Michielsen\( ^{*,\dag} \)}\\
    {Second Examiner: Prof. Dr. Holger Rauhut\( ^* \)}\\
    {Adviser: Dr. Dennis Willsch\( ^\dag \)\vspace{0.1cm}}\\
    {\footnotesize\( ^* \)RWTH Aachen University, D-52056 Aachen, Germany}\\
    {\footnotesize\( ^\dag \)Jülich Supercomputing Centre, Institute for Advanced Simulation,}\\
    {\footnotesize Forschungszentrum Jülich, D-52425 Jülich, Germany\vspace{0.5cm}}
}
\date{Date TBD}


\begin{document}
\maketitle
\pagenumbering{roman}

%----------------------------------------------------------------------------------------
% Abstract
%----------------------------------------------------------------------------------------
\clearpage\shipout\null
\chapter*{Abstract}

In this thesis we explore using the D-Wave Advantage 4.1 quantum annealer to sample quantum Boltzmann distributions and train quantum Boltzmann machines (QBMs).
We focus on the real-world problem of using QBMs as generative models to produce synthetic foreign exchange market data and analyze how the results stack up against classical models based on restricted Boltzmann machines.
Additionally, we study a small 12-qubit problem which we use to compare samples obtained from the annealer to theory, and in the process gain vital insights into how well the Advantage 4.1 can sample quantum Boltzmann random variables and be used to train QBMs.
Through this we are able to show that the D-Wave Advantage 4.1 can sample classical Boltzmann random variables to some extent, but is limited in its ability to sample from quantum Boltzmann distributions.
Our findings indicate that models trained using the annealer are much noisier than simulations and struggle to perform at the same level as classical models.

\clearpage\shipout\null

%----------------------------------------------------------------------------------------
% Table of Contents
%----------------------------------------------------------------------------------------
\tableofcontents
\clearpage\shipout\null

%----------------------------------------------------------------------------------------
% Introduction
%----------------------------------------------------------------------------------------
\chapter{Introduction}
\label{ch:introduction}
\pagenumbering{arabic}
In recent years we have seen the inception of cloud-based quantum computing, with a number of different providers offering various services.
In terms of maturity, the quantum computing industry as a whole is still in the early stages and there are a lot of obstacles left to overcome before mainstream adoption.
Quantum computing is not only trying to advance the theory and technology, but also yearning for practical applications in which quantum computing offers advantages over classical computing.

There are two main branches of quantum computing: universal quantum computing, i.e., gate-based quantum computing, and adiabatic quantum computing, i.e., quantum annealing.
In our work here we focus on the latter, as current generation devices are slightly more mature and have much higher numbers of qubits than the former.
We discuss the theory behind quantum annealing later in~\cref{sec:quantum_annealing}.
One such cloud-based quantum computing service is D-Wave's Leap platform~\cite{dwave_leap}, which allows users to access quantum annealers and other solvers from anywhere in the world with an internet connection.

D-Wave is a pioneer in this field, having been researching and developing quantum annealers since 1999.
They revolutionized the field with the release of the world's first commercially available quantum annealer in 2011~\cite{zyga_2011}.
Since then, they have released a new version every 2-3 years, each having more qubits and couplers than the previous.
Their latest version, the D-Wave Advantage, has over 5000 qubits with 15 connections per qubit~\cite{dwave_advantage}.

In this thesis we take a journey into the field of quantum machine learning and explore the possibilities of using quantum Boltzmann machines (QBMs) as generative models for real-world financial data.
As we will see, there is a deep connection between the quantum Boltzmann machine and quantum annealing, allowing one to train QBMs using a quantum annealer.

Risk management is one of the most important components of the financial system, and in 2008 it failed, leading to the financial crisis which wreaked havoc on economies around the world.
The success of risk management hinges on how accurately the underlying risk models capture the true behavior of the market.
Therefore, it is essential that we continuously strive to find new and innovative ways of modeling that can help us understand the real risks involved and implement policies to effectively mitigate such risks.

In the globalized economy of today, foreign exchange (forex) fluctuations expose a number of firms to a lot of risk if not properly mitigated.
Forex markets had a daily volume of \$6.6T in 2019~\cite{bis_2019}, the majority of which was concentrated in a few major pairs.
In the 2019 paper \textit{The Market Generator}~\cite{kondratyev_2019}, Kondratyev and Schwarz detail how a classical restricted Boltzmann machine (RBM) can be used to generate synthetic forex data, and the advantages it offers over traditional parametric models.
We use their work as a basis to build our classical models upon, which we then use as a reference to compare our quantum models with.

In~\cref{ch:data_analysis} we start by visualizing the data set in various ways to get an idea how it is distributed.
We further analyze quantitative metrics to get a better understanding of some of the intricacies of the data set.
Finally, we go through and detail how we preprocess the data set into a model-friendly format.

With the data set in hand, we move to explaining the theory behind the classical RBM in~\cref{ch:rbm}, and describing some of the difficulties associated with training and using classical RBMs.
We then train several classical models on the data set discussed in~\cref{ch:data_analysis} using different preprocessing methods, and compare them with each other using visualizations and a number of quantitative metrics.

In~\cref{ch:qbm} we start from the theory of quantum Boltzmann machines, detailing how they work and their connection to quantum annealing.
We study a small 12-qubit problem which we can simulate, allowing us to compare annealer performance with that of theory, and gaining key insights into how to train and use QBMs.
With those insights, we move to the final stage of training a model using the data set from~\cref{ch:data_analysis}, then assessing the performance versus the classical models from~\cref{ch:rbm}.
Additionally, we cover some of the challenges of using D-Wave quantum annealers to train QBMs in~\cref{sec:challenges}.

Lastly, we summarize our findings in~\cref{ch:conclusion}, as well as discuss future directions in which this research can be expanded.

In addition to all of the research findings, we also introduce the open source Python package \texttt{qbm}~\cite{qbm} to make it easier for the community to train and study quantum Boltzmann machines.
All work presented here is reproducible (except for that involving quantum measurements), and the code is available on the Forschungszentrum J\"ulich Git server~\footnote{\url{https://jugit.fz-juelich.de/qip/qbm-quant-finance}}.


%----------------------------------------------------------------------------------------
% Data Analysis
%----------------------------------------------------------------------------------------
\chapter{Data Analysis \& Preprocessing}
\label{ch:data_analysis}
\section{Data Analysis}
Our raw data set consists of the daily open, high, low, and close (OHLC) values for the time period 1999-01-01 through 2019-12-31 of the following major currency pairs
\begin{itemize}
    \item EURUSD - Euro € / U.S. Dollar \$
    \item GBPUSD - British Pound Sterling £ / U.S. Dollar \$
    \item USDCAD - U.S. Dollar \$ / Canadian Dollar \$
    \item USDJPY - U.S. Dollar \$ / Japanese Yen ¥
\end{itemize}
obtained from Dukascopy historical data feed~\cite{dukascopy}.
We filter the data set to remove days with zero volume, as well as NYSE and LSE holidays, resulting in 5165 training samples.
Here we use the notation \( x_\text{open} \), \( x_\text{high} \), \( x_\text{low} \), and \( x_\text{close} \) to denote the open, high, low, and close values of a currency pair on a particular day.

Given that the raw data values are on an absolute basis, we need to convert them to relative terms in order to be able to compare data from different time periods on a more equal footing.
The natural way to do so is to use the intraday returns
\begin{align}
    r = \frac{x_\text{close} - x_\text{open}}{x_\text{open}}.
\end{align}
However, this is not necessarily the best way to approach this.
Instead, we opt to use the log returns
\begin{align}
    \tilde{r}
        = \log(1+r)
        = \log\bigg( \frac{x_\text{close}}{x_\text{open}} \bigg)
\end{align}
due to several advantages, such as log-normality and small \( r \) approximation~\cite{quantivity_2012}.

We begin our analysis by taking a look at the histograms depicted in~\cref{fig:histograms_raw}.
From visual examination we see that the log returns are roughly normally distributed with the statistics given in~\cref{tbl:data_log_returns_raw_stats}.
\begin{figure}[!htb]
    \begin{center}
        \includegraphics[width=1\linewidth]{data_analysis/histograms.png}
    \end{center}
    \caption{Histograms of the log returns data set.}
    \label{fig:histograms_raw}
\end{figure}

\begin{table}[!htb]
    \centering
    \begin{adjustbox}{max width=\textwidth}
        \input{../tables/data/log_returns_raw_stats.tbl}
    \end{adjustbox}
    \caption{Statistics of the log returns data set.}
    \label{tbl:data_log_returns_raw_stats}
\end{table}

We also visualize the log returns in a violin and box plot in~\cref{fig:violin_raw} to identify outliers and see how they are distributed.
Two major outliers clearly stand out from the rest: one to the downside for the GBPUSD pair, and another to the upside for the USDJPY pair.
The former occurred on 2016-06-24, the day the Brexit referendum result was announced~\cite{brexit_gov_uk}.
The latter occurred on 2008-10-28, right in the midst of the financial crisis when people were talking about the end of the Yen carry trade~\cite{jpy_carry_trade_nyt}.
In the final training data set, we remove outliers greater than \( 10\sigma \) from the mean, resulting in only removing the day corresponding to the Brexit referendum result, which lies \( 11.1\sigma \) below the mean.
\begin{figure}[!htb]
    \begin{center}
        \includegraphics[width=1\linewidth]{data_analysis/violin.png}
    \end{center}
    \caption{Violin and box plot of the log returns data set illustrating the distribution of the outliers.}
    \label{fig:violin_raw}
\end{figure}

Next we examine the correlations between the currency pairs to get an idea of the interdependencies between them.
We visualize this with scatter plots shown in~\cref{fig:scatters} where we observe a clear positive correlation between EURUSD/GBPUSD, and clear negative correlations between EURUSD/USDCAD and GBPUSD/USDCAD, where the / is used to denote the pairs being compared against each other.
This is further verified by the Pearson \( r \), Spearman \( \rho \), and Kendall \( \tau \) correlation coefficients laid out in~\cref{tbl:data_correlation_coefficients}.
Furthermore, we find the correlation coefficients to be positive for pairs of the form \( X \)USD/\( Y \)USD, and negative for pairs of the form \( X \)USD/USD\( Y \), for \( X,Y \in \) \{EUR, GBP, CAD, JPY\}, as expected.
Details on how the correlation coefficients are computed and how to interpret them can be found in~\cref{app:correlation_coefficients}.
\begin{figure}[!htb]
    \begin{center}
        \includegraphics[width=1\linewidth]{data_analysis/scatters.png}
    \end{center}
    \caption{Scatter plots of the log returns data set.}
    \label{fig:scatters}
\end{figure}

\begin{table}[!htb]
    \centering
    \begin{adjustbox}{max width=\textwidth}
        \input{../tables/data/correlation_coefficients.tbl}
    \end{adjustbox}
    \caption{Correlation coefficients of the log returns data set.}
    \label{tbl:data_correlation_coefficients}
\end{table}


\section{Data Preprocessing}
The models in the following chapters require the training data to be in the form of bit vectors, so we must first convert our data set to such a form.
Let \( \mat{X} \in \R^{4 \times N} \) represent the training data set of log returns with \( N \) samples, where training samples are vectors in the column space, thus element \( x_{ij} \) represents the \( i \)th currency pair log return for the \( j \)th training sample.

To discretize the data, we rescale and round the entries of \( \mat{X} \) to integer values in \( \{0, 1, \dots, 2^{n_\text{bits}} - 1\} \), represented by the matrix \( \mat{X}' \in \N^{4 \times N} \) with entries
\begin{align}
    x_{ij}' = \bigg\lfloor \frac{x_{ij} - \min_k \{x_{ik}\}}{\max_k \{x_{ik}\} - \min_k \{x_{ik}\}} \cdot (2^{n_\text{bits}} - 1) \bigg\rceil,
\end{align}
where \( \lfloor \ \cdot \ \rceil \) denotes rounding to the nearest integer.

A new matrix \( \mat{V} \in \binset^{4\cdot n_\text{bits} \times N} \) is then created with the columns being the \( n_\text{bits} \)-length bit vectors corresponding to the binary representation of the entries of the columns of \( \mat{X}' \) concatenated together.
For example, if \( \vec{x}' = (x_1',x_2',x_3',x_4') \) is a column of \( \mat{X}' \) and the function \( \text{bitvector}(x') \) takes in an integer \( x' \) and returns an \( n_\text{bits} \)-bit binary representation bit vector, then the corresponding column in \( \mat{V} \) is
\begin{align}
    \vec{v} = \begin{bmatrix}
        \text{bitvector}(x_1') \\
        \text{bitvector}(x_2') \\
        \text{bitvector}(x_3') \\
        \text{bitvector}(x_4') \\
    \end{bmatrix}
    \in \binset^{4\cdot n_\text{bits}}.
\end{align}

For this research we take \( n_\text{bits} = 16 \), giving us a training set \( \mat{V} \in \binset^{64 \times N} \), thus our training samples are bit vectors of length 64.
The discretization errors associated with this conversion and data set are on the order of \( 10^{-7} \), well within the desired tolerance for this purpose.

\subsection{Data Transformation}\label{sec:outlier_transform}
Due to how the data is linearly converted to a discrete form before rounding, it opens up the possibility of the discretized data being clustered in the mid-range values if large outliers are present.
To mitigate this, we use a transformation to reduce the gap between outliers by scaling outliers beyond a certain threshold \( \tau \) using the procedure detailed in~\cref{alg:transformation}.
We call this the \textit{outlier power transformation}.

In practice, we take \( \tau = 1 \) and \( \alpha = 0.5 \), thus the standardized data points above one standard deviation are mapped to their square roots, as illustrated in~\cref{fig:data_transformation}.
We tested a few other combinations of \( \tau \) and \( \alpha \), but found these values to produce the best model results out of those we tried; of course this could likely be further optimized.
The effect this transformation has on the model results versus the base dataset can be seen in~\cref{sec:rbm_results}.
This transformation is invertible when \( \bar{x} \), \( \sigma_x \), and \( \delta \) are saved.

\begin{algorithm}
\caption{Outlier Power Transformation}
\begin{algorithmic}[1]
    \Procedure{Transform}{$\vec{x}, \alpha, \tau$}
            \Comment $\alpha$ is the power, $\tau$ is the threshold
        \State $N \gets \text{length}(\vec{x})$
        \State $\bar{x} \gets \frac{1}{N} \sum_{i=1}^{N} x_i$
        \State $\sigma_{x} \gets \sqrt{\frac{1}{N} \sum_{i=1}^{N} (x_i - \bar{x})^2}$
        \State $\delta \gets \tau - \tau^\alpha$
            \Comment ensures the transformation is bijective
        \For {$i$ in 1 to $N$}
            \State $x_i \gets (x_i - \bar{x}) / \sigma_x$
                \Comment standardize
            \If {$x_i > \tau$}
                \State $x_i \gets (\abs{x_i}^\alpha + \delta) \cdot \text{sign}(x_i)$
                    \Comment scale standardized values beyond $\tau$
            \EndIf
            \State $x_i \gets x_i \cdot \sigma_x + \bar{x}$
                \Comment undo standardization
        \EndFor
    \EndProcedure
\end{algorithmic}
\label{alg:transformation}
\end{algorithm}

\begin{figure}[!htb]
    \begin{center}
        \includegraphics[width=1\linewidth]{data_analysis/data_transformation.png}
    \end{center}
    \caption{Transformation defined in~\cref{alg:transformation} using \( \tau = 1 \) and \( \alpha = 0.5 \), for the purpose of reducing large gaps in the discretized data set by scaling outliers above \( \tau \) standard deviations.}
    \label{fig:data_transformation}
\end{figure}

Histograms of the transformed data set are shown in~\cref{fig:histograms_transformed}, and a violin and box plot is shown in~\cref{fig:violin_transformed}.
In these, we observe the appearance of "shoulders" around the threshold \( \tau = 1 \) standard deviation, and that the transformed outliers appear much less extreme, allowing us to better utilize the full range of discrete values.
\cref{tbl:data_log_returns_transformed_stats} shows that the transformation reduces the standard deviations to roughly \( 78\% \) of their originals values given in~\cref{tbl:data_log_returns_raw_stats}.

\begin{figure}[!htb]
    \begin{center}
        \includegraphics[width=1\linewidth]{data_analysis/histograms_transformed.png}
    \end{center}
    \caption{Histograms of the outlier power-transformed log returns data set.}
    \label{fig:histograms_transformed}
\end{figure}
\begin{figure}[!htb]
    \begin{center}
        \includegraphics[width=1\linewidth]{data_analysis/violin_transformed.png}
    \end{center}
    \caption{Violin and box plot of the outlier power-transformed log returns data set illustrating the distribution of the rescaled outliers.}
    \label{fig:violin_transformed}
\end{figure}
\begin{table}[!htb]
    \centering
    \begin{adjustbox}{max width=\textwidth}
        \input{../tables/data/log_returns_transformed_stats.tbl}
    \end{adjustbox}
    \caption{Statistics of the outlier power-transformed log returns data set.}
    \label{tbl:data_log_returns_transformed_stats}
\end{table}

\subsection{Additional Information}
As mentioned in~\cite{kondratyev_2019}, one can use additional binary indicator variables to enrich the training data set.
One such bit of information is the rolling volatility relative to the historical median (see~\cref{app:annualized_volatility} for definition of annualized volatility).
If the 3-month rolling volatility is below (above) the historical median it is assigned a value of 0 (1) to indicate the low (high) volatility regime.
The 3-month rolling volatilities versus their historical medians are plotted in~\cref{fig:rolling_volatility}.

These additional binary indicator variables are then concatenated onto the training data set and fed to the model to make it more flexible by allowing for the model outputs to be conditioned on a specific volatility regime.
Adding one indicator for each of the four currency pairs increases the number of rows in our training data set by four, thus the volatility-concatenated data set is in the space \( \binset^{68 \times N} \).

\begin{figure}[!htb]
    \begin{center}
        \includegraphics[width=1\linewidth]{data_analysis/rolling_volatility.png}
    \end{center}
    \caption{3-month rolling volatilities of the log returns data set compared with their historical medians.}
    \label{fig:rolling_volatility}
\end{figure}


%----------------------------------------------------------------------------------------
% RBM
%----------------------------------------------------------------------------------------
\chapter{Restricted Boltzmann Machine}
\label{ch:rbm}
\section{Theory}
The Restricted Boltzmann Machine (RBM) is an energy-based model defined by the energy function
\begin{align}
    E(\vec{v}, \vec{h})
        &= -\sum_i a_i v_i - \sum_j b_j h_j - \sum_{i,j} v_i w_{ij} h_j \label{eq:rbm_energy} \\
        &= -\vec{a}^\intercal\vec{v} - \vec{b}^\intercal\vec{h} - \vec{v}^\intercal\mat{W}\vec{h} \label{eq:rbm_energy_vectorized}
\end{align}
where
\begin{itemize}
    \item \( \vec{v} \in \Z_2^{n_v} \) represents the visible units, with associated bias vector \( \vec{a} \in \R^{n_v} \).
    \item \( \vec{h} \in \Z_2^{n_h} \) represents the hidden units, with associated bias vector \( \vec{b} \in \R^{n_h} \).
    \item \( \mat{W} \in \R^{n_v \times n_h} \) represents the weights corresponding to the interaction strengths between visible and hidden units.
\end{itemize}

\begin{figure}[ht]
    \begin{center}
        \includegraphics[width=1\linewidth]{rbm_diagram.png}
    \end{center}
    \caption{The structure of a restricted Boltzmann machine with \( n_v \) visible units and \( n_h \) hidden units.}
    \label{fig:rbm_diagram}
\end{figure}

It is termed "restricted" due to the fact that intra-layer connections are not allowed, i.e., visible units are only connected to hidden units, and vice versa.
An example is depicted in \cref{fig:rbm_diagram}.
We also Note that in this case we use RBM to refer to what one might call a Bernoulli RBM, due to the fact that the visible and hidden units are Bernoulli random variables.

The probability to find the system in the configuration \( \{\vec{v},\vec{h}\} \) is given by the Boltzmann distribution (with \( \beta = \frac{1}{k_BT} = 1 \) )
\begin{align}
    p(\vec{v}, \vec{h}) = \frac{1}{Z} e^{-E(\vec{v},\vec{h})}
\end{align}
with intractable~\cite{long_servedio_2010} partition function
\begin{align}
    Z = \sum_{\vec{v},\vec{h}} e^{-E(\vec{v},\vec{h})}
\end{align}

Due to the independence of units in the same layer, the conditional probabilities of the layers are given by~\footnote{Here \( \sigma(x) \) is the logistic sigmoid function and \( \odot \) denotes element-wise multiplication.} (see \cref{app:conditional_probabilities_derivation} for full derivation)
\begin{align}
    p(\vec{h} | \vec{v})
        &= \prod_j \sigma\big( (2\vec{h} - 1) \odot (\vec{b} + \mat{W}^\intercal\vec{v}) \big)_j \\
    p(\vec{v} | \vec{h})
        &= \prod_i \sigma\big( (2\vec{v} - 1) \odot (\vec{a} + \mat{W}\vec{h}) \big)_i
\end{align}

Due to the partition function being intractable, the model able to be solved analytically, thus one must resort to other methods to optimize it such as likelihood maximization.
Rather than maximizing the likelihood function itself, it is common to minimize the negative log-likelihood.
For data distribution \( p_\text{data} \) and parameters \( \theta = (\mat{W}, \vec{a}, \vec{b}) \) the average log-likelihood is given by
\begin{align}
    \ell(\theta) = \sum_{\vec{v}} p_{\text{data}}(\vec{v}) \log \sum_\vec{h} \frac{1}{Z} e^{-E(\vec{v},\vec{h})}
\end{align}
with gradients (see \cref{app:rbm_log_likelihood_derivation} for full derivation)
\begin{align}
    \partial_{w_{ij}} \ell(\theta)
        &= \langle v_i h_j \rangle_{\text{data}} - \langle v_i h_j \rangle_{\text{model}} \\
    \partial_{a_i} \ell(\theta)
        &= \langle v_i \rangle_{\text{data}} - \langle v_i \rangle_{\text{model}} \\
    \partial_{b_j} \ell(\theta)
        &= \langle h_j \rangle_{\text{data}} - \langle h_j \rangle_{\text{model}}
\end{align}
Therefore, the parameters at step \( t \) are given by
\begin{align}
    \mat{W}^{(t)}
        &= \mat{W}^{(t-1)} + \eta(\langle \vec{v} \vec{h}^\intercal \rangle_{\text{data}} - \langle \vec{v} \vec{h}^\intercal \rangle_{\text{model}}) \\
    \vec{a}^{(t)}
        &= \vec{a}^{(t-1)} + \eta(\langle \vec{v} \rangle_{\text{data}} - \langle \vec{v} \rangle_{\text{model}}) \\
    \vec{b}^{(t)}
        &= \vec{b}^{(t-1)} + \eta(\langle \vec{h} \rangle_{\text{data}} - \langle \vec{h} \rangle_{\text{model}})
\end{align}
where \( \eta \) is the learning rate hyperparameter.

Since \( p(\vec{v}) \) can not be sampled directly, it must be sampled using a Markov chain Monte Carlo (MCMC) method.
The way to do so is through Gibbs sampling, which uses the conditional probabilities \( p(\vec{h}|\vec{v}) \) and \( p(\vec{v}|\vec{h}) \).
One starts with a visible vector and then samples the hidden units, followed by sampling the visible units, and so forth until a desired thermalization threshold is reached.
How many steps it requires to reach thermalization is model dependent, and can be estimated by analyzing the autocorrelations of a Markov chain.
The algorithm for Gibbs sampling is given in \cref{alg:Gibbs} and illustrated in \cref{fig:gibbs_sampling_diagram}.
For brevity the algorithm is presented in a vectorized format.

\begin{algorithm}
\caption{Gibbs Sampling}
\begin{algorithmic}[1]
    \Procedure{Gibbs}{$\vec{v},n,\mat{W},\vec{a},\vec{b}$}
        \State $n_v \gets$ length$(\vec{a})$
        \State $n_h \gets$ length$(\vec{b})$
        \For{$k$ in 1 to $n$}
            \State $\vec{r} \sim$ Uniform$(0, 1, n_h)$
            \State $\vec{h} \gets \vec{r} < \sigma(\vec{b} + \mat{W}^\intercal\vec{v})$
                \Comment $\sigma, <$ applied element-wise
            \State $\vec{r} \sim$ Uniform$(0, 1, n_v)$
            \State $\vec{v} \gets \vec{r} < \sigma(\vec{a} + \mat{W}\vec{h})$
                \Comment $\sigma, <$ applied element-wise
        \EndFor
        \State \Return $\vec{v}$
    \EndProcedure
\end{algorithmic}
\label{alg:Gibbs}
\end{algorithm}
The Uniform$(a, b, n)$ function in \cref{alg:Gibbs} produces a length \( n \) vector of uniform i.i.d. random variables on the interval $[a, b)$.

\begin{figure}
    \begin{center}
        \includegraphics[width=1\linewidth]{gibbs_sampling_diagram.png}
    \end{center}
    \caption{Illustration of the Gibbs sampling procedure.}
    \label{fig:gibbs_sampling_diagram}
\end{figure}

The standard procedure for training an RBM is called \( n \)-step contrastive divergence (CD-\( n \)), with \( n \) often taken as 1 in practice.
The algorithm is detailed in \cref{alg:CDn}, where one can see that \( n \) corresponds to how many Gibbs sampling steps are between the positive and negative phase gradient estimates.
Applying the algorithm to a mini-batch is essentially the same except except that one divides the learning rate by the size of the mini-batch to get a mini-batch averaged gradient estimate.

\begin{algorithm}
    \caption{$n$-Step Contrastive Divergence (CD-$n$)}
\begin{algorithmic}[1]
    \Procedure{CD}{$\vec{v}_+,n,\mat{W},\vec{a},\vec{b},\eta$}
        \Comment $\vec{v}_+$ is a training example
        \State $\vec{h}_+ \gets \sigma(\vec{b} + \mat{W}^\intercal\vec{v}_+)$
            \Comment $\sigma$ applied element-wise
        \State $\vec{v}_- \gets$ Gibbs$(\vec{v}_+,n,\mat{W},\vec{a},\vec{b})$
        \State $\vec{h}_- \gets \sigma(\vec{b} + \mat{W}^\intercal\vec{v}_-)$
            \Comment $\sigma$ applied element-wise
        \State $\mat{W} \gets \mat{W} + \eta(\vec{v}_+ \vec{h}_+^\intercal - \vec{v}_- \vec{h}_-^\intercal)$
        \State $\vec{a} \gets \vec{a} + \eta(\vec{v}_+ - \vec{v}_-)$
        \State $\vec{b} \gets \vec{b} + \eta(\vec{h}_+ - \vec{h}_-)$
        \State \Return $\mat{W}, \vec{a}, \vec{b}$
    \EndProcedure
\end{algorithmic}
\label{alg:CDn}
\end{algorithm}


%\section{Application}

%\section{Results}
%In this section we will compare four different RBM models, denoted as:
%\begin{itemize}
    %\item (B): baseline model.
    %\item (V): using volatility indicators.
    %\item (X): using a transformed feature space.
    %\item (XV): using a transformed feature space and volatility indicators.
%\end{itemize}

%\begin{table}[ht]
    %\centering
    %\begin{adjustbox}{max width=\textwidth}
        %\input{../tables/rbm/correlation_coefficients.tbl}
    %\end{adjustbox}
    %\caption{Correlation coefficients of the data vs. samples generated by the RBM models. The RBM numbers are shown in the format average \(\pm\) 1 standard deviation from an ensemble of size 100.}
    %\label{tbl:rbm_correlation_coefficients}
%\end{table}

%\begin{table}[ht]
    %\centering
    %\begin{adjustbox}{max width=\textwidth}
        %\input{../tables/rbm/qq_rmses.tbl}
    %\end{adjustbox}
    %\caption{QQ root mean squared errors of the RBM models. All numbers are shown in the format average \(\pm\) 1 standard deviation from an ensemble of size 100.}
    %\label{tbl:rbm_qq_rmse}
%\end{table}

%\begin{table}[ht]
    %\centering
    %\begin{adjustbox}{max width=\textwidth}
        %\input{../tables/rbm/volatilities.tbl}
    %\end{adjustbox}
    %\caption{Historical volatilities of the data vs. samples generated by the RBM models. All numbers are shown in the format average \(\pm\) 1 standard deviation from an ensemble of size 100.}
    %\label{tbl:rbm_volatilities}
%\end{table}

%\begin{table}[ht]
    %\centering
    %\begin{adjustbox}{max width=\textwidth}
        %\input{../tables/rbm/conditional_volatilities.tbl}
    %\end{adjustbox}
    %\caption{Conditional historical volatilities of the data vs. samples generated by the RBM models. All numbers are shown in the format average \(\pm\) 1 standard deviation from an ensemble of size 100.}
    %\label{tbl:rbm_conditional_volatilities}
%\end{table}

%\begin{table}[ht]
    %\centering
    %\begin{adjustbox}{max width=\textwidth}
        %\input{../tables/rbm/autocorrelation_times.tbl}
    %\end{adjustbox}
    %\caption{Integrated autocorrelation times of the RBM models.}
    %\label{tbl:rbm_ac_times}
%\end{table}

%\begin{table}[ht]
    %\centering
    %\begin{adjustbox}{max width=\textwidth}
        %\input{../tables/rbm/tails.tbl}
    %\end{adjustbox}
    %\caption{Lower and upper tails, i.e., 1st and 99th percentiles, of the data vs. samples generated by the RBM models. All numbers are shown in the format average \(\pm\) 1 standard deviation from an ensemble of size 100.}
    %\label{tbl:rbm_tails}
%\end{table}

%\begin{figure}[ht]
    %\begin{center}
        %\includegraphics[width=1\linewidth]{rbm/tail_concentrations.png}
    %\end{center}
    %\caption{Tail concentration functions of the data vs. samples generated by the RBM models.}
    %\label{fig:rbm_tail_concentrations}
%\end{figure}

%\begin{figure}[ht]
    %\begin{center}
        %\includegraphics[width=1\linewidth]{rbm/autocorrelation_functions.png}
    %\end{center}
    %\caption{Autocorrelation functions of the RBM models. Function values are computed from Gibbs sample chains of length \( 10^8 \).}
    %\label{fig:rbm_autocorrelation_functions}
%\end{figure}

%\begin{figure}[ht]
    %\begin{center}
        %\includegraphics[width=1\linewidth]{rbm/qq.png}
    %\end{center}
    %\caption{QQ plots of the RBM models.}
    %\label{fig:rbm_qq_plots}
%\end{figure}


%----------------------------------------------------------------------------------------
% QBM
%----------------------------------------------------------------------------------------
\chapter{Quantum Boltzmann Machine}
\label{ch:qbm}
\section{Theory}
\todo{Quantum annealing theory}

\todo{Explain QUBO}

\section{Application}
\todo{D-Wave introduction/usage}

\section{Results}


%----------------------------------------------------------------------------------------
% Conclusion
%----------------------------------------------------------------------------------------
\chapter{Conclusion}
\label{ch:conclusion}
\section{Summary}
We started with an analysis of the forex log returns data set in~\cref{ch:data_analysis}, analyzing the data from a number of aspects to get an understanding of the intricacies.
After that we moved to training classical RBM models in~\cref{ch:rbm}, where we were able to produce good results similar to those in~\cite{kondratyev_2019}.
The outlier power transformation we detailed in~\cref{sec:outlier_transform} showed much promise; models trained on the transformed data sets performed noticeably better than those trained on the base data sets.

After establishing a classical baseline to compare our quantum models with, we moved to studying a small 12-qubit problem in~\cref{sec:qbm_12_qubit_problem}, through which we gained a deeper understanding of how to sample quantum Boltzmann random variables using a D-Wave quantum annealer, specifically the Advantage 4.1.
There we were able to match sample distributions returned by the annealer to distributions from the family of distributions corresponding to the density operator \( \rho(s,T) = \frac{1}{Z}e^{-\beta H(s)} \).
Our findings indicated that, in general, the Advantage 4.1 was not able to sample from just any quantum Boltzmann distribution, rather just those that were classical Boltzmann-like in nature.
To be more specific, not only did the results match a single distribution, but rather a subset of the family that satisfied \( B(s) / T = \text{constant} \), as indicated by the streak patterns observed in~\cref{fig:dkl_min_heatmap}.
This occurs when the distribution is similar to one late in the anneal process, i.e., when \( e^{-\beta H(s^*)} \approx e^{-\beta B(s^*) H_\text{final}} \).
This was likely due to the annealer not being able to quench the system fast enough, allowing for nontrivial dynamics to occur, as the shortest allowed quench durations were still on the order of a few hundred nanoseconds, which is still quite long when considering that the qubits oscillate at a frequency in terms of gigahertz.
How closely the annealer could approximate a desired classical Boltzmann distribution was found to be dependent on both the embedding as well as the anneal schedule, thus it is highly recommended to tune these accordingly.

With the information that only classical Boltzmann distributions could be sampled well, we moved to training a bound-based quantum restricted Boltzmann machine (BQRBM), which essentially reduced the problem to a classical RBM trained using quantum assistance.
The difficulty of choosing the effective temperature was in this case easily circumvented by treating \( \beta \) as a learnable parameter as described in~\cref{sec:learning_beta} and verified in~\cref{sec:qbm_simulation_results}.
We trained BQRBM models using both a simulation and the Advantage 4.1 annealer, allowing us to compare exactly how close the annealer trained model was to the theory.
Additionally, we trained classical RBM to use as a reference point.
In short, the BQRBM model trained using the quantum annealer underperformed both the classical RBM as well as the simulation, as seen in~\cref{sec:qbm_annealer_results}.
The simulation-based model showed promise though, outperforming the classical RBM, offering hope for future annealer-trained models if annealers can further reduce the information loss associated with sampling (quantum) Boltzmann distributions.

Finally, we used the knowledge gained about how to train a small BQRBM and applied it to training a larger one in~\cref{sec:quantum_market_generator} using the log returns data set, mapping 94 logical qubits to 398 physical qubits with chain lengths of up to 7.
This model proved to be more challenging to train because setting the annealer hyperparameters (chain strength, anneal schedule, and embedding) could not be done as in the 12-qubit problem due to the fact that we could not simulate such a large system.
In practice we had to choose these values by doing a limited hyperparameter scan, which was difficult due to increased training times that averaged around 15 minutes per epoch.
Longer epoch times originated from a combination of solver load and latency from Europe to the North American west coast.
This meant that training a model for 100 epochs would have taken around a day, and if we wanted to fully train the models for all of the hyperparameters in our scan it would have taken weeks.

The results in~\cref{sec:qbm_log_returns_results} illustrated that the BQRBM was able to learn to produce synthetic data according to the log returns data set distribution to some extent, but drastically underperformed the classical RBM.
This could likely have been improved with a more exhaustive hyperparameter scan, but that was not necessarily feasible given the time requirements, and it is unclear if the results would have been drastically better given that even the 12-qubit BQRBM trained on the annealer was unable to outperform the classical RBM.

In this thesis we laid out a framework with which one can train quantum Boltzmann machines using both simulation and D-Wave quantum annealers.
As part of this thesis, the Python package \texttt{qbm}~\cite{qbm} was developed to make it easy to train and study QBMs.
This package is open source and available to the public to encourage further study of QBMs.

Overall, this thesis furthered not only our understanding of QBMs, but that of D-Wave annealer sampling in general.
We hope that this work will be useful for future research and development.


\section{Future Directions}
Throughout this thesis we came across several directions which we would have liked to explore more in depth but did not have the time to.

It would be interesting to investigate if adding technical indicators to the log returns data set could increase model performance.
Given that the log returns data set only took into account the currency pairs' behavior over one day (excluding the volatility indicators), technical indicators calculated using data over a historical window could enrich the data set with vital information to help the model better learn the complexities of the distribution.

The discretization procedure for converting continuous data into bit vectors could probably be further improved.
As we saw in~\cref{sec:classical_market_generator}, the models that used the outlier power-transformed data sets had lower KL divergences and better reproduced the correlations between the currency pairs.

Of most interest is simulating the time-dependent Schr\"odinger equation of the D-Wave annealer to determine how fast the system needs to quench in order to freeze out the dynamics.
This would give us a good idea of how much quantum annealers need to improve in order to be able to sample from truly quantum Boltzmann distributions.

Studying additional anneal schedule formats would also be a very interesting direction.
Reverse annealing was tested to a very small extent here only to see if it produced drastically different results than forward annealing, but was left out of the final research because the results did not show any significant improvements and led to added complexity due to the need to choose what state the system was initialized in and whether or not the system was reinitialized to the same state after each measurement.


%----------------------------------------------------------------------------------------
% Appendix
%----------------------------------------------------------------------------------------
\appendix
\chapter{Appendix}
\label{ch:appendix}
\section{Restricted Boltzmann Machine}
\subsection{Conditional Probabilities}\label{app:conditional_probabilities_derivation}
This derivation follows along the lines of that found on p. 658-659 of~\cite{goodfellow_deep_learning}.
We start by noting
\begin{align}
    p(\vec{v}) = \frac{1}{Z} \sum_\vec{h} e^{-E(\vec{v},\vec{h})}
\end{align}
From this we can derive the conditional probability
\begin{align}
\begin{split}
    p(\vec{h} | \vec{v})
        &= \frac{p(\vec{v},\vec{h})}{p(\vec{v})} \\
        &= \frac{1}{p(\vec{v})} \frac{1}{Z} \exp( \vec{a}^\intercal\vec{v} + \vec{b}^\intercal\vec{h} + \vec{v}^\intercal\mat{W}\vec{h} ) \\
        &= \frac{1}{Z'} \exp\bigg( \sum_j b_j h_j + \sum_j (\vec{v}^\intercal\mat{W})_j h_j \bigg) \\
        &= \frac{1}{Z'} \prod_j \exp\big( b_j h_j + (\vec{v}^\intercal\mat{W})_j h_j \big)
\end{split}
\end{align}
where
\begin{align}
    Z' = \sum_\vec{h} \exp( \vec{b}^\intercal\vec{h} + \vec{v}^\intercal\mat{W}\vec{h} )
\end{align}
This leads us to
\begin{align}
\begin{split}
    p(h_j = 1 | \vec{v})
        &= \frac{\tilde{p}(h_j = 1 | \vec{v})}{\tilde{p}(h_j = 0 | \vec{v}) + \tilde{p}(h_j = 1 | \vec{v})} \\
        &= \frac{\exp\big( b_j + (\vec{v}^\intercal\mat{W})_j \big)}{1 + \exp\big( b_j + (\vec{v}^\intercal\mat{W})_j \big)} \\
        &= \sigma\big( b_j + (\vec{v}^\intercal\mat{W})_j \big)
\end{split}
\end{align}
Finally, we have
\begin{align}
    p(\vec{h} | \vec{v}) = \prod_j \sigma\big( (2\vec{h} - 1) \odot (\vec{b} + \mat{W}^\intercal\vec{v}) \big)_j
\end{align}

Analogously for \( p(\vec{v} | \vec{h}) \) one finds
\begin{align}
    p(\vec{v} | \vec{h}) = \prod_i \sigma\big( (2\vec{v} - 1) \odot (\vec{a} + \mat{W}\vec{h}) \big)_i
\end{align}

\subsection{Log-Likelihood}\label{app:rbm_log_likelihood_derivation}
For data distribution \( p_\text{data} \) and parameters \( \theta = (\mat{W}, \vec{a}, \vec{b}) \) the average log-likelihood is given by
\begin{align}
\begin{split}
    \ell(\theta)
        &= \sum_{\vec{v}} p_{\text{data}}(\vec{v}) \log p(\vec{v}) \\
        &= \sum_{\vec{v}} p_{\text{data}}(\vec{v}) \log \sum_\vec{h} p(\vec{v},\vec{h}) \\
        &= \sum_{\vec{v}} p_{\text{data}}(\vec{v}) \log \bigg(\frac{1}{Z} \sum_\vec{h} e^{-E(\vec{v},\vec{h})}\bigg) \\
        &= \sum_{\vec{v}} p_{\text{data}}(\vec{v}) \log \sum_\vec{h} e^{-E(\vec{v},\vec{h})} - \log \sum_{\vec{v},\vec{h}} e^{-E(\vec{v},\vec{h})}
\end{split}
\end{align}
Taking the gradient we find
\begin{align}
\begin{split}
    \partial_{\theta} \ell(\theta)
        &= \sum_{\vec{v}} p_{\text{data}}(\vec{v}) \frac{\sum_\vec{h} e^{-E(\vec{v},\vec{h})} \partial_{\theta}\big( -E(\vec{v},\vec{h}) \big) }{\sum_\vec{h} e^{-E(\vec{v},\vec{h})}}
            - \frac{\sum_{\vec{v},\vec{h}} e^{-E(\vec{v},\vec{h})} \partial_{\theta}\big( -E(\vec{v},\vec{h}) \big) }{\sum_{\vec{v},\vec{h}} e^{-E(\vec{v},\vec{h})}} \\
        &= \sum_{\vec{v}} p_{\text{data}}(\vec{v}) \Big\langle \partial_{\theta}\big( -E(\vec{v},\vec{h}) \big) \Big\rangle_{p(\vec{h}|\vec{v})}
        - \Big\langle \partial_{\theta}\big( -E(\vec{v},\vec{h}) \big) \Big\rangle_{p(\vec{v},\vec{h})}
\end{split}
\end{align}
We will use \( \langle \ \cdot \ \rangle_{\text{data}} \) to denote the expectation value w.r.t.\ the data, and \( \langle \ \cdot \ \rangle_{\text{model}} \) to denote the expectation value w.r.t.\ the model.
This gives us
\begin{align}
\begin{split}
    \partial_{w_{ij}} \ell(\theta)
        &= \langle v_i h_j \rangle_{\text{data}} - \langle v_i h_j \rangle_{\text{model}} \\
    \partial_{a_i} \ell(\theta)
        &= \langle v_i \rangle_{\text{data}} - \langle v_i \rangle_{\text{model}} \\
    \partial_{b_j} \ell(\theta)
        &= \langle h_j \rangle_{\text{data}} - \langle h_j \rangle_{\text{model}}
\end{split}
\end{align}

\section{Quantum Boltzmann Machine}
\subsection{Log-Likelihood}\label{app:qbm_log_likelihood_derivation}
This derivation follows along the lines of that laid out in~\cite{amin_2018}.
We start with the data-averaged log-likelihood
\begin{align}
\begin{split}
    \ell(\theta)
        &= \sum_{\vec{v}} p_{\text{data}}(\vec{v}) \log p(\vec{v}) \\
        &= \sum_{\vec{v}} p_{\text{data}}(\vec{v}) \log \frac{\tr(\Lambda_\vec{v} e^{-H})}{\tr(e^{-H})} \\
        &= \sum_{\vec{v}} p_{\text{data}}(\vec{v}) \Big[ \log\tr(\Lambda_\vec{v} e^{-H}) - \log\tr(e^{-H}) \Big]
\end{split}
\end{align}
Taking the partial derivative yields
\begin{align}
    \label{eq:qbm_log_likelihood_derivative}
    \partial_\theta \ell(\theta)
        &= \sum_{\vec{v}} p_{\text{data}}(\vec{v}) \bigg[ \frac{\tr(\Lambda_\vec{v} \partial_\theta e^{-H})}{\tr(\Lambda_\vec{v} e^{-H})} - \frac{\tr(\partial_\theta e^{-H})}{\tr(e^{-H})} \bigg]
\end{align}
Due to the noncommutativity of \( H \) and \( \partial_\theta H \), we need to use the trick laid out in~\cite{amin_2018} where we take \( e^{-H} = (e^{-\delta\tau H})^n \) with \( \delta\tau \equiv 1 / n \), which allows one to write
\begin{align}
    \partial_\theta e^{-H}
        &= -\sum_{m=1}^{n} e^{-m\delta\tau H} \delta\tau \partial_\theta He^{-(n-m)\delta\tau H} + \mathcal{O}(\delta\tau^2)
\end{align}
Taking the limit as \( n \rightarrow \infty \) of both sides gives
\begin{align}
\begin{split}
    \partial_\theta e^{-H}
        &= \lim_{n\rightarrow\infty} -\sum_{m=1}^{n} e^{-m\delta\tau H} \delta\tau \partial_\theta He^{-(n-m)\delta\tau H} + \mathcal{O}(\delta\tau^2) \\
        &= -\int_{0}^{1} d\tau e^{-\tau H} \partial_\theta H e^{(\tau-1)H}
\end{split}
\end{align}
From here one can take the trace of both sides to arrive at
\begin{align}
\begin{split}
    \tr(\partial_\theta e^{-H})
        &= -\tr\bigg( \int_{0}^{1} d\tau e^{-\tau H} \partial_\theta H e^{(\tau-1)H} \bigg) \\
        &= -\int_{0}^{1} d\tau \tr\big(e^{-\tau H} \partial_\theta H e^{(\tau-1)H} \big) \\
        &= -\int_{0}^{1} d\tau \tr\big(e^{(\tau-1)H} e^{-\tau H} \partial_\theta H \big) \\
        &= -\int_{0}^{1} d\tau \tr\big(e^{-H} \partial_\theta H \big) \\
        &= -\tr\big(e^{-H} \partial_\theta H \big)
\end{split}
\end{align}
Which gives
\begin{align}
\begin{split}
    \frac{\tr(\partial_\theta e^{-H})}{\tr(e^{-H})}
        &= -\frac{\tr(e^{-H} \partial_\theta H)}{\tr(e^{-H})} \\
        &= -\tr(\rho \partial_\theta H) \\
        &= -\langle \partial_\theta H \rangle
\end{split}
\end{align}
Unfortunately, due to the additional factor of \( \Lambda_\vec{v} \) in the first term of \cref{eq:qbm_log_likelihood_derivative}, one arrives at
\begin{align}
\begin{split}
    \tr(\Lambda_\vec{v} \partial_\theta e^{-H})
        &= -\tr\bigg( \int_{0}^{1} d\tau \Lambda_\vec{v} e^{-\tau H} \partial_\theta H e^{(\tau-1)H} \bigg) \\
        &= -\int_{0}^{1} d\tau \tr\big(\Lambda_\vec{v} e^{-\tau H} \partial_\theta H e^{(\tau-1)H} \big)
\end{split}
\end{align}
which is nontrivial to compute in practice.

\subsection{Log-Likelihood Lower Bound}\label{app:qbm_log_likelihood_lower_bound}
This derivation follows along the lines of that laid out in~\cite{amin_2018}.
The Golden-Thompson inequality that \( \tr(e^{A}e^{B}) \ge \tr(e^{A+B}) \) allows one to write (for small \( \epsilon > 0 \))
\begin{align}
    \tr(e^{-H} e^{\log(\Lambda_\vec{v}+\epsilon)}) \ge \tr(e^{-H+\log(\Lambda_\vec{v}+\epsilon)})
\end{align}
Taking the limit \( \epsilon \rightarrow 0 \) yields
\begin{align}
    \tr(\Lambda_\vec{v}e^{-H}) \ge \tr(e^{-H_\vec{v}})
\end{align}
where
\begin{align}
    H_\vec{v} &= \braket{\vec{v} | H | \vec{v}}
\end{align}
with \( H_\vec{v} \) being the "clamped" Hamiltonian.
This is called clamped because the visible qubits are held to the classical state of the visible vector \( \vec{v} \) due to an infinite energy penalty imposed by the \( \log(\Lambda_\vec{v} + \epsilon) \) term.
Using this we can write the inequality
\begin{align}
\begin{split}
    p(\vec{v})
        &= \frac{\tr(\Lambda_\vec{v} e^{-H})}{\tr(e^{-H})} \\
        &\ge \frac{\tr(e^{-H_\vec{v}})}{\tr(e^{-H})}
\end{split}
\end{align}
which in turn allows for the log-likelihood to be bounded as
\begin{align}
    \ell(\theta) \ge \tilde{\ell}(\theta)
\end{align}
where
\begin{align}
    \tilde{\ell}(\theta)
        &\equiv \sum_\vec{v} p_\text{data}(\vec{v}) \log\frac{\tr(e^{-H_\vec{v}})}{\tr(e^{-H})}
\end{align}
\subsection{Log-Likelihood Lower Bound Derivative}\label{app:qbm_log_likelihood_lower_bound_derivative}
This derivation follows along the lines of that laid out in~\cite{amin_2018}.
Taking the partial derivative yields
\begin{align}
\begin{split}
    \label{eq:qbm_log_likelihood_derivative_lower_bound}
    \partial_\theta \tilde{\ell}(\theta)
        &= \sum_{\vec{v}} p_{\text{data}}(\vec{v}) \bigg[ \frac{\tr(\partial_\theta e^{-H_\vec{v}})}{\tr(e^{-H_\vec{v}})} - \frac{\tr(\partial_\theta e^{-H})}{\tr(e^{-H})} \bigg] \\
        &= \sum_{\vec{v}} p_{\text{data}}(\vec{v}) \bigg[ \frac{\tr(e^{-H_\vec{v}} \partial_\theta H_\vec{v})}{\tr(e^{-H_\vec{v}})} - \frac{\tr(e^{-H} \partial_\theta H)}{\tr(e^{-H})} \bigg] \\
        &= \sum_{\vec{v}} p_{\text{data}}(\vec{v}) [ \tr(\rho_\vec{v} \partial_\theta H_\vec{v}) - \tr(\rho \partial_\theta H) ] \\
        &= \sum_{\vec{v}} p_{\text{data}}(\vec{v}) [ \langle \partial_\theta H_\vec{v} \rangle_\vec{v} - \langle \partial_\theta H \rangle ] \\
        &= \overline{\langle \partial_\theta H_\vec{v} \rangle_\vec{v}} - \langle \partial_\theta H \rangle
\end{split}
\end{align}
Plugging in our parameters we get
\begin{align}
\begin{split}
    \partial_{w_{ij}} \tilde{\ell}(\theta)
        &= \overline{\langle \sigma_i^z \sigma_j^z \rangle_\vec{v}} - \langle \sigma_i^z \sigma_j^z \rangle \\
        &= \langle \sigma_i^z \sigma_j^z \rangle_\text{data} - \langle \sigma_i^z \sigma_j^z \rangle_\text{model} \\
    \partial_{b_i} \tilde{\ell}(\theta)
        &= \overline{\langle \sigma_i^z \rangle_\vec{v}} - \langle \sigma_i^z \rangle \\
        &= \langle \sigma_i^z \rangle_\text{data} - \langle \sigma_i^z \rangle_\text{model}
\end{split}
\end{align}
When restrictions are imposed on connections within the hidden layer, the clamped Hamiltonian reduces to
\begin{align}
    H_\vec{v}
        &= -\sum_i \big(\Gamma_i \sigma_i^x + b_i'(\vec{v}) \sigma_i^z\big)
\end{align}
where \( b_i'(\vec{v}) = b_i + (\mat{W}^\intercal\vec{v})_i \).
This allows one to rewrite the clamped density matrix as
\begin{align}
\begin{split}
    \rho_\vec{v}
        &= \frac{1}{Z_\vec{v}} \exp\bigg( \sum_i \big(\Gamma_i \sigma_i^x + h_i'(\vec{v}) \sigma_i^z\big) \bigg) \\
        &= \frac{1}{Z_\vec{v}} \prod_i \exp \big(\Gamma_i \sigma_i^x + b_i'(\vec{v}) \sigma_i^z\big) \\
        &= \prod_i \rho_\vec{v}^{(i)}
\end{split}
\end{align}
With this we can compute the expectation values as
\begin{align}
\begin{split}
    \langle \sigma_i^z \rangle_\vec{v}
        &= \tr(\rho_\vec{v}^{(i)}\sigma_i^z) \\
        &= \frac{\tr\bigg[ \exp \big(\Gamma_i \sigma_i^x + b_i'(\vec{v}) \sigma_i^z\big) \sigma_i^z \bigg]}{\tr\bigg[ \exp \big(\Gamma_i \sigma_i^x + b_i'(\vec{v}) \sigma_i^z\big) \bigg]} \\
        &= \frac{b_i'(\vec{v})}{D_i(\vec{v})} \tanh\big(D_i(\vec{v})\big)
\end{split}
\end{align}
where \( D_i(\vec{v}) = \sqrt{\Gamma_i^2 + b_i'(\vec{v})^2} \).

The last equality above is obtained by using that for traceless \( A \) with \( \det A < 0 \) we can write
\begin{align}
    \exp(A) = \cosh\Big(\sqrt{\abs{\det A}}\Big) I + \frac{1}{\sqrt{\abs{\det A}}}\sinh\Big(\sqrt{\abs{\det A}}\Big) A
\end{align}
This is obtained by using Cayley-Hamilton theorem along with the series expansion of the matrix exponential and grouping the terms.

\subsection{Learning The Effective Inverse Temperature \( \beta \)}\label{app:learning_beta}
Following along the lines of~\cite{xu_2021} we have
\begin{align}
\begin{split}
    p_\text{DW}
        &= \frac{1}{Z_\text{DW}} e^{-E_\text{DW}} \\
        &= \frac{1}{Z_\text{DW}} e^{-E / \betahat} \\
\end{split}
\end{align}
which leads to a negative log-likelihood derivative of
\begin{align}
    -\frac{\partial \log p_\text{DW}}{\partial\betahat}
        &= -\frac{1}{\betahat^2} (E - \langle E \rangle)
\end{align}
and after averaging over the training data configurations we have
\begin{align}
    \Delta\betahat
        &= \frac{\eta}{\betahat^2}\big(\langle E \rangle_\text{data} - \langle E \rangle_\text{model}\big)
\end{align}

\section{Model Implementation in Python}
A Python package named \texttt{qbm} was developed to house all of the reusable code for this project.
The code implementing the quantum restricted Boltzmann machine, trained by minimizing the upper bound of the negative log-likelihood, is implemented as an easy to use Python class titled \texttt{BQRBM}.
All code is commented and broken down into easy to understand methods and blocks.

\subsection{Initialization}
The object can be initialized as follows
\begin{python}
model = BQRBM(
    X_train,
    n_hidden,
    embedding,
    anneal_params,
    beta_initial=1.5,
    beta_range=(0.1, 10),
    qpu_params={"region": "na-west-1", "solver": "Advantage_system4.1"},
    exact_params=None,
    seed=0,
)
\end{python}
where
\begin{itemize}
    \item \texttt{X\_train (np.ndarray)}: training data set of visible vectors, of shape \newline\texttt{(n\_training\_examples, n\_visible)}.
    \item \texttt{n\_hidden (int)}: number of hidden units.
    \item \texttt{embedding (dict)}: embedding dict (can be generated by \texttt{minorminer}).
    \item \texttt{anneal\_params (dict)}: dict containing the keys \texttt{s} corresponding to \( s^* \), \texttt{A} corresponding to \( A(s^*) \), and \texttt{B} corresponding to \( B(s^*) \) that determine which distribution the model will use to approximate with, and \texttt{anneal\_schedule} which is a list of tuples of the form \( (t, s) \) defining the anneal schedule.
    \item \texttt{beta\_initial (float)}: the initial value of \( \betahat \) to approximate the effective \( \beta \). It is recommended to tune this accordingly. One method is to train a model for a couple of epochs, see where the value settles around, then set to that on the next initialization.
    \item \texttt{beta\_range (2-tuple of floats)}: range of allowed values for \( \betahat \). Used for clipping \( \betahat \) to sane values, e.g. to avoid accidentally overshooting and learning a negative value.
    \item \texttt{qpu\_params (dict)}: parameters dict to initialize the qpu.
    \item \texttt{exact\_params (dict)}: optional dict with keys \texttt{beta} corresponding to the effective \( \beta \) the exact sampler will use. If not \texttt{None} will result in the model using the exact sampler rather than the annealer.
    \item \texttt{seed (int)}: seed for the (pseudo) random number generator used in shuffling the mini-batches, as well as the exact sampler when used.
\end{itemize}

\subsection{Training}
The model can be trained using the \texttt{train()} method
\begin{python}
model.train(
    n_epochs=100,
    learning_rate=1e-1,
    learning_rate_beta=1e-1,
    mini_batch_size=10,
    n_samples=10_000,
    callback=None,
)
\end{python}
where
\begin{itemize}
    \item \texttt{n\_epochs (int)}: the number of epochs to train.
    \item \texttt{learning\_rate (float or list of floats)}: the learning rate to use when updating the weights and biases. If a \texttt{list} then must be of length \texttt{n\_epochs}, where \texttt{learning\_rate[i-1]} is the learning rate to use at epoch \texttt{i}. It is highly recommended to tune this accordingly.
    \item \texttt{learning\_rate\_beta (float or list of floats)}: the learning rate to use when updating \( \betahat \). If a \texttt{list} then must be of length \texttt{n\_epochs}, where \newline\texttt{learning\_rate\_beta[i-1]} is the learning rate to use during epoch \texttt{i}. It is highly recommended to tune this accordingly.
    \item \texttt{mini\_batch\_size (int)}: size of the mini-batches. It is recommended to keep this around 10, similar to a classical RBM.
    \item \texttt{n\_samples (int)}: number of samples to generate at the end of every epoch to use for updating \( \betahat \), as well as in the \texttt{callback()} function. It is recommended to keep this as \( 10^4 \) when using the annealer to get the best read on the statistics as possible.
    \item \texttt{callback (function)}: function which takes the arguments \texttt{(model, sample\_state\_vectors)} to call at the end of every epoch for the purpose of getting a read on the model's training progress. The \texttt{model} argument is the BQRBM object, and the \texttt{sample\_state\_vectors} are the sampled state vectors returned by the sampler, of shape \texttt{(n\_samples, n\_qubits)} with the first \texttt{n\_visible} qubits corresponding to the visible units. If \texttt{None}, then no callback function will be used.
\end{itemize}

\subsubsection{Training Loop}
Below we have a code snippet from the core of the \texttt{train} method
\begin{python}[breaklines=true]
for epoch in range(1, n_epochs + 1):
    # loop over the mini-batches and compute and apply grads for each
    for mini_batch_indices in self._random_mini_batch_indices(mini_batch_size):
        V_pos = self.X_train[mini_batch_indices]
        self._compute_positive_grads(V_pos)
        self._compute_negative_grads(V_pos.shape[0])
        self._apply_grads(self.learning_rate / V_pos.shape[0])
        self._clip_weights_and_biases()

    # generate samples after every epoch to update beta
    samples = self.sample(n_samples)
    self._update_beta(samples)

    # pass samples to the callback function to track performance
    if callback is not None:
        callback_output = callback(self, self._get_state_vectors(samples))
        self.callback_outputs.append(callback_output)
\end{python}
In the above training loop we see the \texttt{\_clip\_weights\_and\_biases()} method being called.
This method clips the weights and biases so that the corresponding \( h_i \) and \( J_{ij} \) values fall within the allowed ranges for the annealer, which for the Advantage 4.1 is \( h_i \in [-4, 4] \) and \( J_{ij} \in [-1, 1] \)~\cite{dwave_solver_properties}.
In principle, it is undesirable to have the weights and biases ever be clipped as it alters the state of the problem.
Therefore, this is just here so that the annealer only ever receives valid \( h_i \) and \( J_{ij} \) values.

\subsection{Sampling}
After the model is trained it can be sampled via the \texttt{sample()} method
\begin{python}
model.sample(
    n_samples,
    answer_mode="raw",
    use_gauge=True,
    binary=False,
)
\end{python}
where
\begin{itemize}
    \item \texttt{n\_samples (int)}: size of the sample set to be generated. The D-Wave Advantage 4.1 annealer is limited to a maximum of \( 10^4 \).
    \item \texttt{answer\_mode (str)}: either \texttt{"raw"} or \texttt{"histogram"}, see the corresponding D-Wave documentation~\cite{dwave_solver_parameters} for more information.
    \item \texttt{use\_gauge (bool)}: if \texttt{True} will use a random gauge transformation, the results will be automatically transformed back to the original space. This is usually recommended to mitigate any readout biases the QPU might have.
    \item \texttt{binary (bool)}: if \texttt{True} will return the sampled state vectors in binary values rather than spin eigenvalues.
\end{itemize}

\subsection{Saving and Loading}
Models can easily be saved and loaded via the corresponding \texttt{save} and \texttt{load} methods.
A model can be saved using
\begin{python}
model.save(file_path)
\end{python}
and loaded again using
\begin{python}
model = BQRBM.load(file_path)
\end{python}

\section{Miscellaneous}
\subsection{Annualized Volatility}\label{app:annualized_volatility}
In finance, the annualized volatility of a time series vector \( \vec{x} \) is computed as
\begin{align}
    \text{vol}(\vec{x}) = \sqrt{252} \cdot \text{std}(\vec{x})
\end{align}
where the factor of \( \sqrt{252} \) comes from the square root of the number of trading days in a year, i.e., it's the annualization factor.

\subsection{KL Divergence}\label{app:kl_divergence}
The Kullback-Leibler divergence~\cite{kullback_1951} is measure of how much the probability distribution \( q \) differs from the reference probability distribution \( p \).
It is defined as
\begin{align}
    \DKL{p}{q}
        &= \sum_{x} p(x) \log\frac{p(x)}{q(x)}
\end{align}
It can be interpreted as the amount of information loss associated with using \( q \) to approximate \( p \).
It must also be noted that the KL divergence is a distance, but not a metric (rather a divergence), because of the asymmetry that \( \DKL{p}{q} \ne \DKL{q}{p} \).

\subsubsection{KL Divergence in Practice}\label{app:kl_divergence_in_practice}
If we wish to compare the distribution \( \psamples \) obtained from real world sampling to that of the reference distribution \( \ptheory \) computed exactly, we need to first do some preprocessing to ensure that our comparison carries some sort of meaning.
Since \( \psamples \) is derived from a finite number of observations we will not always be able to get a full read of the true distribution.
Therefore, it is ideal to take a histogram approach to better approximate the shape of the distribution and compare it to the same binned shape of the reference distribution.
In practice this depends on how many samples ones has, and how many samples there are in comparison to the total number of states of the distribution.

For the problem of comparing the distribution obtained by sampling the D-Wave annealer (maximum \( 10^4 \) sample size) in relation to \( 2^{12} = 4096 \) total possible configurations, we chose the number of bins in the histogram to be 32, since this is close to the number of bins computed using the Freedman-Diaconis rule on some of the sample sets.

An issue we run into in practice is that \( \psamples \) computed via the histogram approach might contain zeros, which leads to issues when computing the KL divergence since it would diverge to \( \infty \) due to the zero in the denominator of the log argument of
\begin{align}
    \DKL{\ptheory}{\psamples} = \sum_i \ptheory^{(i)} \log\frac{\ptheory^{(i)}}{\psamples^{(i)}}
\end{align}
The way around this is the concept of smoothing~\cite{han_kl_divergence}, i.e., adding a small probability to the zeros in the case where we know the true probability to be nonzero.
In this case, given the quantum nature of the problem and the distribution we know that no state has a truly zero probability (although they can be infinitesimally small).
The real question is what probability do we add here so that the numbers still carry some meaning.
For this we chose to take the following approach: if \( \psamples^{(i)} = 0 \) and \( \ptheory^{(i)} > 0 \), then we take \( \psamples^{(i)} = \ptheory^{(i)} \cdot 10^{-6} \), which means that the term \( \ptheory^{(i)} \log \frac{\ptheory^{(i)}}{\psamples^{(i)}} = \ptheory^{(i)} \log 10^6 \).
Thus, any zeros in the sample distribution that are nonzero in the exact distribution will impose a \( \ptheory \) scaled constant penalty to the KL divergence.
The sum of the terms of the sample probabilities that used to be zero are then evenly subtracted from the rest of the probabilities to ensure that the distribution remains normalized.

Even though this is not the true KL divergence as intended by theory, it provides a decent approximation for which we can use to compare distributions generated by real world processes with limited sample size.


\subsection{Correlation Coefficients}\label{app:correlation_coefficients}
The Pearson correlation coefficient is defined as
\begin{align}
    \rho_{X,Y} = \frac{\text{cov}{(X,Y)}}{\sigma_X \sigma_Y} \in [-1, 1]
\end{align}
and measures the linear correlation between the random variables \( X \) and \( Y \).
Therefore, it must be noted that this does not capture nonlinear relations, thus it shouldn't be relied upon to tell the full story.
Additionally, this measure is quite sensitive to outliers.

The Spearman rank correlation coefficient is defined as
\begin{align}
    r_s = \rho_{R(X),R(Y)} = \frac{\text{cov}{\big(R(X),R(Y)\big)}}{\sigma_{R(X)} \sigma_{R(Y)}} \in [-1, 1]
\end{align}
and is the Pearson correlation coefficient of the rank of the random variables \( X \) and \( Y \).
The main difference to the Pearson correlation coefficient is that the Spearman measures the monotonic relationship, regardless of linearity.
The Spearman correlation coefficient is also less sensitive to outliers than the Pearson.

The Kendall rank correlation coefficient is defined as
\begin{align}
    \tau = \frac{2}{n(n-1)} \sum_{i<j} \text{sign}(x_i - x_j) \text{sign}(y_i - y_j) \in [-1, 1]
\end{align}
where \( (x_1, y_1), \dots, (x_n, y_n) \) are pairs of observations of the random variables \( X \) and \( Y \).

It is important to keep in mind how one interprets the correlation coefficients.
The sign of the correlation coefficient determines whether or not it is negatively or positively correlated, and the magnitude determines how strong the correlation effects are.
As a loose guide, correlation coefficient values of 0.1, 0.3, and 0.5 can be termed small, medium, and large, respectively~\cite{research_design_and_statistical_analysis}.
In general, one must be careful when interpreting the correlation coefficients, thus it is important to understand what the values mean, and what they don't.
Section 3.4.2 "Interpreting the Correlation Coefficient" of~\cite{research_design_and_statistical_analysis} offers further insight and points out some of the pitfalls to watch out for.

In this thesis the correlation coefficients are computed using the respective functions from the SciPy Python package~\cite{python_scipy}.

\subsection{Tail Concentration Functions}\label{app:tail_concentration_functions}
The lower tail concentration function is defined as~\cite{venter_2002}
\begin{align}
\begin{split}
    L(z)
        &= \frac{p(U_1 \le z, U_2 \le z)}{z} \\
        &= \frac{C(z,z)}{z}
\end{split}
\end{align}
and the upper as
\begin{align}
\begin{split}
    R(z)
        &= \frac{p(U_1 > z, U_2 > z)}{1-z} \\
        &= \frac{1 - 2z + C(z,z)}{1-z}
\end{split}
\end{align}
where \( U_1 \) and \( U_2 \) uniform random variables on the interval \( [0, 1] \), and \( C(u_1, u_2) \) is the copula of \( (U_1, U_2) \).

In practice we compute \( U_1 \) and \( U_2 \) as the normalized rank of the observations of the random variables \( X \) and \( Y \), respectively.
The way to interpret the concentration functions is that they represent the probability that \( X \) and \( Y \) simultaneously take on extreme values.
When plotted, the lower tail concentration function is used for \( 0 \le z \le 0.5 \) and the upper for \( 0.5 \le z \le 1 \).

\subsection{Autocorrelation Analysis}\label{app:autocorrelation_analysis}
When studying results from an MCMC-based model it is important to be aware that sequentially generated samples are not always statistically independent, that is, there is some thermalization threshold that corresponds to the minimum number of sampling steps between samples to consider them as statistically independent.

For a time series \( x_1, \dots, x_n \), the lag-\( k \) autocorrelation function is defined as~\cite{time_series_analysis}
\begin{align}
    \rho_k
        &= \frac{\text{cov}(x_t, x_{t+k})}{\sigma_{x}^2}
\end{align}
The autocorrelation function is essentially a correlation coefficient, except instead of comparing two different variables it compares the same variable at different times.
In this thesis we use the statsmodels Python package~\cite{python_statsmodels} to compute the autocorrelation function as there are some caveats when computing it in practice with large chains (e.g. there are some tricks such as using a Fourier transformation to make the computations more efficient).

The integrated autocorrelation time is a reasonable estimate of how many steps in between samples we should have before we can consider them to be (to a degree) statistically independent.
In this thesis we use the emcee Python package~\cite{python_emcee} to estimate the integrated autocorrelation time, which follows the approach laid out by Goodman and Weare in~\cite{goodman_weare_2010}.

\subsection{Exact Computation of \( \rho \)}\label{app:exact_rho_computation}
For the density matrix
\begin{align}
    \rho = \frac{1}{Z} e^{-\beta H}
\end{align}
we can compute this as
\begin{align}
    \rho
        &= \frac{1}{\tr(A)} S A S^{-1}
\end{align}
where
\begin{align}
    A = \text{diag}\Big(e^{-\beta(\lambda_1 - \lambda_{\min})}, \dots, e^{-\beta(\lambda_{2^n} - \lambda_{\min})}\Big)
\end{align}
with \( \lambda_i \) being the eigenvalues of \( H \), and \( S \) is the matrix of eigenvectors that transforms \( H \) to and from its eigenspace.
The reason why we subtract \( \lambda_{\min} \) from the eigenvalues in practice is to avoid computing the exponential of a large number which can lead to divergence in floating point calculations.

\subsection{Learning Rate Decay Schedule}\label{app:lr_exp_decay}
The learning rate at epoch \( t \) is given by
\begin{align}
    \eta^{(t)}
        &= \eta^{(0)} \cdot \min\bigg\{1, 2^{-\frac{t - t_\text{decay}}{T_\text{decay}}}\bigg\}
\end{align}
where \( \eta^{(0)} \) is the initial learning rate, \( t_\text{decay} \) is the epoch at which the decay begins, and \( T_\text{decay} \) is the decay period.


%----------------------------------------------------------------------------------------
% References
%----------------------------------------------------------------------------------------
\printbibliography

\end{document}
